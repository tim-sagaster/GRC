\documentclass[a4paper, 11pt]{article}

\usepackage[utf8]{inputenc}
\usepackage{amsmath, amssymb}
\usepackage[ngerman]{babel}
\usepackage{parskip}
\usepackage{graphicx}
\usepackage{tikz}
\usepackage{mathtools}
\usepackage{empheq}

\numberwithin{equation}{section}
%\renewcommand{\vec}[1]{\boldsymbol{#1}}

\newcommand{\ubtext}[2]{\underbrace{#1}_{\mathclap{\text{#2}}}}

\begin{document}

\title{Introduction to General Relativity and Cosmology\\\small{Wolfgang Schweiger WS2019/2020}}
\author{Erik Kraml \\ Tim Sagaster}
\maketitle
\newpage
\tableofcontents
\newpage


\setcounter{section}{-1}
\section{Einleitung und Motivation}
\subsection*{Einführende Zitate}
\begin{enumerate}
\item W.Pauli ''The general theory of relativity ....  beauty in its mathematical structure.''
\item M.Born: ''(The gerneral theory of relativity) seemed and still seems ... admire it as a work of art.''
\item A.Einstein (to A.Sommerfeld) ''At present I occupy myself ... the original relativity is child's play.''

\end{enumerate}

\subsection*{Abkürzungen}
SR ... spezielle Relativitätstheorie \\
ART ... allgemeine Relativitätstheorie - relativistische Theorie der Gravitation\\
E-GL ... Einstein-GLeichung\\
IS ... Inertial System

\subsection*{Einführung}
\begin{itemize}
\item Äquivalenzprinzip: Gravitationseffekte lassen sich durch Übergang zu geeignet beschleunigtem Bezugssystem wegtransformieren (z.B. Satellitenlabor, ...)

\item Kovarianzprinzip: Physikalische Gleichungen sollen als Tensorgleichungen formuliert werden, da sie in allen Bezugssytemen dieselbe Form aufweisen.
\end{itemize}

Einstein erkannte, dass man Kräfte die durch Gravitation auftreten in dem metrischen Tensor absorbieren kann. Die Metrik hängt dadurch vom betrachteten Raumzeitpunkt ab. Dies entspricht einer gekrümmten Manigfaltigkeit, daher spricht man von ''gekrümmter Raum-Zeit''. In diesem Raum muss der kürzeste Weg zwischen zwei Punkten nicht mehr unbedingt die Gerade sein, sondern kann ein Bogen sein.

\subsection*{Einstein-Gleichung}
Beschreibt Zusammenhang zwischen metrischem Tensor und Massenverteilung im Raum.
\begin{equation*}
g_{\mu \nu}(x) \leftrightarrow \text{Massenverteilung im Raum}
\end{equation*}
$g_{\mu \nu}(x)$ $\dots$ Raum-Zeit abhängiger metrischer Tensor\\
Lösung der E-GL für feste Massenverteilung:

\renewcommand{\labelitemi}{$\rightarrow$}
\begin{itemize}
\item Metrik in der Nähe eines massiven Sterns\\
physikalische Konsequenzen: 
\begin{itemize}
\item Lichtableitung in der Nähe eier großen Masse
\item gravitative Rotverschiebung
\item ART Effekte auf Periheldrehung (insbesondere Merkur) 
\end{itemize}

\item Gravitationswellen durch beschleunigte Massen \\
LIGO-Experiment (2015)
\item Sternentwicklung\\
\begin{tikzpicture}
\node (A) at (0,0) {Gaswolke};
\node (B) at (3,0) {Supernova};
\node at (7, -1) {Schwarzes Loch};
\node at (7, 1) {Neutornenstern};
\node (C) at (5.5,-1) {};
\node (D) at (5.5,1) {};
\draw[->] (A) -- (B);
\draw[->] (B) -- (C);
\draw[->] (B) -- (D);
\end{tikzpicture}

\item Geschichte des Universums \\
Kosmologisches Standardmodell (Robertson-Walker-Metrik)\\
$R(t) \xrightarrow{t \rightarrow 0} 0$ : Singularität der Lösung (Big Bang)\\
3K kosmische Hintergrundstrahlung
\end{itemize}

\subsection{Von Newtons Gravitationstheorie zu den \\Einstein-Gleichungen}
\underline{Ziel:} relative Verallgemeinerung von Newtons Gravitationstheorie\\
1687: ''Philosophie naturalis principia mathematica''\\
$N$ Massepunkte die gravitativ wechselwirken: 
\begin{equation}
\label{glg:0_Newton}
m_i \dfrac{g^2\vec{r}_i(t)}{dt^2} = - G \sum^N_{\substack{i,j = 1\\ i \neq j}} \dfrac{m_i m_j (\vec{r} _i(t)-\vec{r}_j(t^\prime))}{|\vec{r}_i(t) - \vec{r}_j(t)|^3}
\end{equation}

$\vec{r}_i(t)$ ... Position des $i$-ten Teilchen\\
$m_i$ ... Masse des $i$-ten Teilchen 
\begin{equation}\label{glg:0_Gravitationskonstante}
G = (6.67408 \pm 0.00031)\times 10^{-11} \dfrac{m^3}{kg s^2}
\end{equation}
$G$ ... Gravitationskonstante (2014 CODATA)

\subsubsection*{Gravitationspotential}
$m:= m_i$, $\vec{r}(t):= \vec{r}_i(t)$ :
\begin{equation}
\label{glg:0_Gravitationspotential}
\Phi(\vec{r}) = - G \sum_{j \neq i} \dfrac{m_j}{|\vec{r}(t)-\vec{r}_j(t)|} = -G \int d^3r^\prime \dfrac{\rho(\vec{r}^\prime)}{|\vec{r}-\vec{r}^\prime|}
\end{equation}
\begin{equation}
\label{glg:0_Massendichte}
\rho(\vec{r}^\prime)=\sum\limits_{j\neq i} m_j \delta^{(3)}(\vec{r}^\prime-\vec{r})
\end{equation}
$\rho(\vec{r}^\prime)$ ... Massendichte


Dynamische Gleichung für Massen:
\begin{equation}
\label{glg:0_dynamischeGleichungFuerMassen}
\boxed{
\underbrace{m}_{\text{träge Masse}} \dfrac{d^2\vec{r}(t)}{dt^2} = -\underbrace{m}_{\mathclap{\text{schwere Masse}}} \vec{\nabla} \Phi(\vec{r}(t))
}
\end{equation}

Feldgleichung für Gravitationspotential:
\begin{equation}
\label{glg:0_FeldgleichungGravitationspotential}
\boxed{
\Delta \Phi = 4 \pi G \rho(\vec{r})
}
\end{equation}
Vergleich Electrostatik:
\begin{equation}
\label{glg:0_LadungImStatischenElektrischenFeld}
m \dfrac{d^2 \vec{r}(t)}{dt^2} = - q \vec{\nabla} \Phi_{el}(\vec{r})
\end{equation}
$\Phi_{el}$ ... elektrisches Potential

\begin{equation}
\label{glg:0 FeldgleichungElektrischesFeld}
\Delta \Phi_{el}(\vec{r}) = 4 \pi \rho_{el}(\vec{r})
\end{equation}
$\rho_{el}$ ... elektrische Ladungsdichte

%%----------------------------------------14.10.-----------------------------------------

\subsubsection*{Relativistische Verallgemeinerung der Elektrostatik $\rightarrow$ Elektrodynamik} 
\begin{equation}
\label{glg:0_verallLaplace}
\Delta \rightarrow \Box = \dfrac{1}{c^2} \dfrac{\partial^2}{\partial t^2} - \Delta
\end{equation}

\begin{equation}
\label{glg:0_verallLadungsdichte}
\rho_{el} \rightarrow (\rho_{el}c, \rho_{el}\vec{v}) = (j^\alpha) \text{      } \alpha = 0,1,2,3
\end{equation}

\begin{equation}
\label{glg:0_verallElektrischesFeld}
\Phi_{el} \rightarrow (\Phi_{el}, \vec{A}) = (A^\alpha) \text{      } \alpha = 0,1,2,3
\end{equation}
$A^\alpha$ ... elektrodynamische Potentiale

\begin{equation}
\label{glg:0_verallElektrischeFeldgleichung}
\Delta \Phi(t, \vec{r}) = - 4 \pi \rho_{el}(\vec{r}) \rightarrow \Box \underbrace{A^\alpha(t, \vec{x})}_{\mathclap{\text{forminvariant bei relativistischen Transformationen zwischen verschiedenen ISen}}} = \dfrac{4 \pi}{c} j^\alpha(t, \vec{x}) \text{      } \alpha = 0,1,2,3
\end{equation}
Maxwell-Gleichungen für elektrodynamische Potentiale in Lorentz-Eichung $\partial_\alpha A^\alpha = 0$

\subsubsection*{Relativistische Verallgemeinerung der Gravitation}
\renewcommand{\labelitemi}{$\bullet$}
\begin{itemize}
\item $\Delta \rightarrow \Box$

\item \ref{glg:0_verallLadungsdichte} kann nicht direkt übernommen werden.\\
Gesamteladung ist eine erhaltene Größe, hängt \underline{nicht} vom Bewegungszustand der einzelnen Teilchen ab!
\begin{equation}
\overbrace{\partial_\alpha j^\alpha = 0}^{\mathclap{\text{Stromerhaltung}}} \Rightarrow \overbrace{\int_{t=const}d^3rj^0 = \int_{t=const}d^3rc \rho_{el}(\vec{r}) = Q}^{\text{Ladungserhaltung}}
\end{equation}
\underline{ABER:} Gesatmasse hängt vom Bewegungszustand der einzelnen Teilchen ab.\\
Erhaltungsgröße $\rightarrow$ gesamte invariante Masse des Systems:
\begin{equation}
\underbrace{M^2}_{\mathclap{\text{unabhängig von IS}}} = E^2 - \vec{P^2} \qquad \vec{P} = \sum_{i=1}^N \vec{p}_i \qquad E= \sum_{i=1}^N \sqrt{m_i^2+\vec{p}_i^2}
\end{equation}

Energie Impuls Tensor:
\begin{equation*}
(T^{\alpha\beta}) =  \begin{pmatrix}
\rho c^2 & \rho c \vec{v}^T \\
\rho c \vec{v} & \rho \vec{v} \vec{v}^T
\end{pmatrix}
\end{equation*}
\begin{equation}
\partial_\beta T^{\alpha\beta} = 0 \Rightarrow \dfrac{1}{c} \int_{t=konst.}d^3r T^{\alpha 0} = P^\alpha \qquad \alpha = 0,1,2,3
\end{equation}
''Ladungserhaltung'' 

\begin{equation}
\rho \rightarrow \begin{pmatrix}
\rho c^2 & \rho c \vec{v}^T \\
\rho c \vec{v} & \rho \vec{v} \vec{v}^T
\end{pmatrix} \sim T^{\alpha \beta}
\end{equation}


\begin{equation}
\Delta \Phi = 4 \pi \rho \rightarrow \boxed{\Box \underbrace{g^{\alpha \beta}}_{\mathclap{\text{metrischer Tensor}}} \sim G T^{\alpha\beta}   }
\end{equation}
\end{itemize}




\newpage
\section{Spezielle Relativitätstheorie}
Bezugssystem\\
räumliche Koordinaten + Uhr (Zeitkoordinate)
ausgezeichnete Bezugssystem: Inertialsysteme (ISe)


\subsubsection*{Galilei'sches Relativitätsprinzip}
\renewcommand{\labelitemi}{-}
\begin{itemize}
\item Alle ISe sind gleichwertig $\rightarrow$ physikalische Gesetze besitzen in allen ISen diesselbe Form (forminvariant bzw. kovariant)
\item newtons Axiome gelten in allen ISen
\end{itemize}
Allgemeinste Übergänge zwischen ISen $\rightarrow$ \underline{Galilei-Transformationen}
\begin{equation}
\begin{aligned}
IS &\rightarrow IS^\prime \\
\vec{x} &\rightarrow \vec{x}^\prime = \hat{O} \vec{x} + \vec{a} + \vec{v}t \\
t &\rightarrow t^\prime = t + t_0
\end{aligned}
\end{equation}


$\hat{O}^{-1} =  \hat{O}^T$ ... Orthogonale Rotationsmatrix

\renewcommand{\labelitemi}{$\bullet$}
\begin{itemize}
\item Translation in $t$ um $t_0$
\item Translation in $\vec{x}$ um $\vec{a}$
\item Rotation von $\vec{v}$ mit $\hat{O}$
\item Boost mit $\vec{v}$
\end{itemize}

Galilei-transformation $\rightarrow$ Probleme mit IS-Unabhängigkeit der \\ Lichtgeschwindigkeit (im Vakuum)
$c = 299792458 m s^{-1}$

\subsubsection*{Einstein'sches Relativitätsprinzip}
\renewcommand{\labelitemi}{-}
\begin{itemize}
\item Alle ISe sind gleichwertig
\item Die Maxwell'schen Gleichungen gelten in allen ISen
\end{itemize}
4-dimensionale Raum-Zeit

MISSING GRAFIK

''Ereignis'' ... Punkt in der 4-dim Raum-Zeit\\
$\left(ct over \vec{x}\right) $ ... ''Ortsvektor'' des Ereignisses
\begin{equation}
\begin{pmatrix}
ct \\
\vec{x}
\end{pmatrix} = \begin{pmatrix}
ct \\ x^1 \\ x^2 \\ x^3
\end{pmatrix} = \begin{pmatrix}
x^0 \\ x^1 \\ x^2 \\ x^3
\end{pmatrix} = x^\alpha  = x \qquad \alpha = 0,1,2,3
\end{equation} 
 
\subsubsection*{Schreibweise:}
\begin{itemize}
\item 4-dim Vektor $x = \begin{pmatrix}
x^0 \\ x^1 \\ x^2 \\x^3 \end{pmatrix} = \begin{pmatrix}
x^\alpha
\end{pmatrix}$ ohne Vektorpfeil
\item 3-dim Vektor $\vec{x} = \begin{pmatrix}
x^1 \\x^2 \\ x^3
\end{pmatrix}$ mit Vektorpfeil
\item Griechische Indizes: $\underbrace{\alpha, \beta, \gamma, ...}_{\mathclap{\text{SRT}}} , \underbrace{\mu, \nu, \rho, ...}_{\mathclap{\text{ART}}}$ bezeichnen Komponenten von 4er Vektoren 
\item Lateinishce Indizes: $i,j,k,... = 1,2,3$ bezeichnen Komponenten von 3-dim Vektoren
\item Indizes hochgestellt $\rightarrow$ kontravariante Vektorkomponenten
\item Indizes tiefgestellt $\rightarrow$ kovariante Vektorkomponenten
\end{itemize}


MISSING GRAFIK\\
Bewegung durch 4-dim Raum-Zeit $\rightarrow$ Abfolg von Ereignissen $\rightarrow$ Weltline


\begin{equation}
\label{glg:1_wegelement}
ds^2 := c^2dt - d\vec{x}^2 = c^2d{t^\prime}^2 - d{\vec{x}^{\prime 2}}
\end{equation}
''Wegelement'' in 4-dim Raum-Zeit unabhängig von IS.
\begin{equation}
\left| \dfrac{d\vec{x}^\prime}{dt^\prime} \right| = c \Rightarrow c^2dt^{\prime 2}-d\vec{x}^{\prime 2} = 0 \Rightarrow c^2dt^2 - d\vec{x}^2 = 0 \xRightarrow{\ref{glg:1_wegelement}} \left|\dfrac{d\vec{x}}{dt} \right| = c
\end{equation}

\subsubsection*{Galilei-Boost}

\begin{equation}
\begin{aligned}
&c =\left| \dfrac{d\vec{x}^\prime}{dt^\prime}\right| = \left| \dfrac{d\vec{x}^\prime}{dt}\right| = \left| \dfrac{d(\vec{x}-\vec{v}t)}{dt}\right| = \left| \dfrac{d\vec{x}}{dt} - \vec{v}\right| \leq \left| \dfrac{d\vec{x}}{dt}\right| + \left| \vec{v} \right| \\
\Rightarrow &\text{ i.a. } \left| \dfrac{d\vec{x}}{dt}\right| \neq c
\end{aligned}
\end{equation}

%%----------------------------------------21.10.-----------------------------------------
Im normalen euklidishcen Raum:
\begin{equation}
d \vec{x} = (dx^1)^2 +(dx^2)^2 + (dx^3)^2 = \begin{pmatrix}
dx^1 \\ dx^2 \\ dx^3 \end{pmatrix} \cdot \begin{pmatrix}
dx^1 \\ dx^2 \\ dx^3 \end{pmatrix} = \begin{pmatrix}
dx^1 & dx^2 & dx^3 \end{pmatrix} \cdot \begin{pmatrix}
dx^1 \\ dx^2 \\dx^3\end{pmatrix}
\end{equation}

\begin{equation}
\begin{aligned}
ds^2 &= \underbrace{c^2dt^2}_{(dx^0)^2} - (dx^1)^2 - (dx^2)^2 - (dx^3)^2 \\
&= \underbrace{\begin{pmatrix} dx^0 & dx^1 & dx^2 & dx^3 \end{pmatrix}}_{dx^T} \underbrace{\begin{pmatrix}
1 & 0 & 0 &0\\ 0 & -1 & 0 & 0 \\ 0 & 0 & -1 & 0 \\ 0& 0 & 0 & -1\end{pmatrix}}_{\eta} \underbrace{\begin{pmatrix} dx^0 \\ dx^1 \\ dx^2 \\dx^3 \end{pmatrix}}_{dx} \\
&= dx^T\eta dx = \sum^3_{\alpha,\beta = 0} dx^\alpha \eta_{\alpha, \beta} \underset{\mathclap{\text{Einstein'sche Summenkonvention}}}{=} dx^\alpha \eta_{\alpha \beta} dx^\beta
\end{aligned}
\end{equation}
$\eta = (\eta_{\alpha \beta})$ ... metrischer Tensor


\subsubsection*{Abstand zwischen zwei beliebigen Punkten}
AB in 4-dimensionaler Raum-Zeit:
\begin{equation}\label{glg:1_abstand4DimRaumZeit}
S^2_{AB} := (x_A- x_B)^2 = \left(x^0_A -x^0_B\right)^2 - \left(\vec{x}_B - \vec{x}_A\right)^2
\end{equation}

\begin{equation}
S^2_{AB} = \begin{cases}
>0, & \text{Zeitartiger Abstand} \\
=0, & \text{Lichtartiger Abstand} \\
<0, & \text{Raumartiger Abstand}
\end{cases}
\end{equation}

MISSING GRAFIK

raumartige Punkte sind nicht kausal verbunden ($\nexists$ Signal mit $v \leq c$, das die beiden Punkte verbindet) 


\subsubsection*{Definieren des Skalarprodukts}

\begin{equation}
x \cdot y = x^0y^0 - x^1y^1-x^2y^2-x^3y^3 = x^0y^0- \vec{x} \cdot \vec{y}
\end{equation}

\begin{equation}\label{glg:1_skalarprodukt4DimRaumZeit}
x \cdot y = \begin{pmatrix}x^0 & x^1 & x^2 & x^3\end{pmatrix} \cdot
\begin{pmatrix} 1 & 0 & 0 &0\\ 0 & -1 & 0 & 0 \\ 0 & 0 & -1 & 0 \\ 0& 0 & 0 & -1\end{pmatrix} \cdot \begin{pmatrix}
y^0 \\ y^1 \\ y^2 \\y^3\end{pmatrix} = x^T \eta y
\end{equation}


In Komponenten Schreibweise:

\begin{equation}
x \cdot y  = x^T \eta y = \sum^3_{\alpha = 0} \sum^3_{\beta = 0} x^\alpha \eta_{\alpha \beta} y^\beta \underset{\mathclap{\text{Einstein'sche  Summenkonvention}}}{=} x^\alpha \eta_{\alpha \beta} y^\beta
\end{equation}

\subsubsection*{Kovariante Vektorkomponenten}

\begin{equation}
\begin{aligned}
y_\alpha &:= \sum^3_{\beta=0} \eta_{\alpha \beta} y^\beta = \eta_{\alpha \beta} y^\beta\\
&= \eta \cdot \begin{pmatrix}y^0 \\ y^1 \\ y^2 \\ y^3\end{pmatrix} = \begin{pmatrix}y^0 \\ -y^1 \\ -y^2 \\ -y^3\end{pmatrix} 
\end{aligned}
\end{equation}

\begin{equation}
x \cdot y = \sum^3_{\alpha = 0} x^\alpha y_\alpha = x^\alpha y_\alpha
\end{equation}

\begin{itemize}
\item 4-dim Raum-Zeit + Skalarprodukt $\rightarrow$ \underline{Minkowski Raum}
\item Skalarprodukt \ref{glg:1_skalarprodukt4DimRaumZeit} induziert Metrik \ref{glg:1_abstand4DimRaumZeit}
\end{itemize}


Wie sehen allgemeine Übergänge zwischen ISen aus?

Allgemeiner linearer Ansatz (Analog zur Galilei Transformation)
\begin{equation}
\begin{aligned}
\text{IS} &\rightarrow \text{IS}^\prime\\
x &\rightarrow x^\prime = \Lambda x + b
\end{aligned}
\end{equation}
$b$ ... 4-dim Vektor für Raum-Zeit-Transformation\\
$\Lambda$ ... 4-dim Matrix für ''Rotation'' in Raum-Zeit

$\Lambda$, $b$ so wählen, dass \ref{glg:1_wegelement} gewährleistet ist.\\
Für $b$: konstant $\surd$ 

Für räumliche Rotation:
\begin{equation}
Lambda = \begin{pmatrix}
1 & 0& 0& 0 \\
0 \\
0 & & \hat{R} \\
0
\end{pmatrix} \surd
\end{equation}
$\hat{R}$ ... orthogonale Rotationsmatrix ($\hat{R}\hat{R}^T- \hat{R^2} \hat{R} = \hat{1}$)

\begin{itemize}
\item Bemerkung: Allgemeine Transformation der Form\\
$x^\prime: \Lambda x + b$\\
nennt man \underline{Poincaré - Transformation} (wenn sie \ref{glg:1_wegelement} erfüllen)
\begin{equation*}
\begin{rcases*}
\text{det}\hat{R} = 1 \text{ keine Raumspiegelung} \\
\text{det}\Lambda = 1 \text{ keine Zeitspiegelung} 
\end{rcases*} \Rightarrow
\begin{aligned}
 &\text{Eigentliche } \\ &\text{Poincaré-Transformation}
\end{aligned}
\end{equation*}


$b=0 \rightarrow$ \underline{Lorentz- Transformation}
\end{itemize}

Wie sehen Lorentz-Boosts aus?

\begin{equation}
\begin{aligned}
dx^\prime = \Lambda dx \\
dx^{\prime \alpha}\Lambda^\alpha_\beta dx^\beta
\end{aligned}
\end{equation}

\begin{equation}
\begin{aligned}
dx^\prime \cdot dx^\prime = dx^{\prime T} \eta dx^\prime = dx^T \underline{\Lambda^T \eta  \Lambda} dx =^!_\text{\ref{glg:1_wegelement}} dx^T \underline{\eta} dx = dx \cdot dx \\
\Rightarrow \boxed{\Lambda \eta \Lambda = \eta}
\end{aligned}
\end{equation}

\begin{equation*}
\left( \Lambda^T \right)_\gamma^\alpha \eta_{\alpha \beta} \Lambda^\beta_\delta = \eta_{\gamma\delta}
\end{equation*}

\begin{equation}
\boxed{
\Lambda^\alpha \gamma \eta_{\alpha \beta} \Lambda^\beta_\delta = \eta_{\gamma\delta}
}
\end{equation}
TEIL A

\begin{equation*}
\begin{aligned}
MISSING
\end{aligned}
\end{equation*}
TEIL B


Für Boost in x-Richtung

\begin{equation}
\Lambda = \left(\Lambda^\alpha_\beta \right) = MISSING MATRIX
\end{equation}

Vernachlässige y- und z- Richtung:
\begin{equation*}
MISSING MATRIX GLEICHUNG
\end{equation*}

\begin{equation*}
\begin{aligned}
\Rightarrow \left(\Lambda^0_0 \right)^2 - \left(\Lambda^1_0 \right)^2 = 1\\
- \left(\Lambda^1_1 \right)^2 + \left(\Lambda^0_1 \right)^2  = -1 \\
\Lambda^0_0 \Lambda^0_1 - \Lambda^1_0 \Lambda^1_1 = 0
\end{aligned}
\end{equation*}

\begin{equation}
\begin{aligned}
\Lambda^0_0 = \cosh \Psi \text{  ,  } \Lambda^1_0 = - \sinh \Psi \text{  1. Gleichung erfüllt} \\
\text{2.Gleichung} \Rightarrow \Lambda ^1_1 = \cosh \Psi \\
\text{3.Gleichung} \Rightarrow \Lambda ^0_1 = -\sinh \Psi
\end{aligned}
\end{equation}

\begin{equation}
MISSING MATRIX GLEICHUNG
\end{equation}

Wie hängt $\Psi$ mit Boostgeschwindigkeit $v$ zusammen?\\
MISSING GRAFIK\\
MISSING GRAFIK

Weltlinie des Ursprungs von IS$^\prime$
\begin{equation*}
MISSING MATRIXGLEICHUNG
\end{equation*}

\begin{equation}
\boxed{
\tanh \Psi = \dfrac{v}{c}}
\end{equation}
$\Psi$ ... ''Rapidität''



$-1 \leq tanh \Psi \leq 1 \Rightarrow \left| \dfrac{v}{c} \right| \leq 1 \Rightarrow c \text{Maximalgeschwindigkeit}$\\
Mit $\cosh^2 \Psi = \dfrac{1}{1-\tanh \Psi}$ und $\sinh^2\Psi = \dfrac{\tanh^2\Psi}{1-\tanh^2 \Psi}$ folgt 

\begin{equation}
\begin{aligned}
MISSING MATRIXGLEICHUNG\\
\boxed{
MISSING MATRIXGLEICHUNG \text{  mit  } \gamma = \cosh \Psi = \dfrac{1}{\sqrt{1- \frac{v^2}{c^2}}}
}
\end{aligned}
\end{equation}

MSSING GRAFIK


Grenzfall $\left( \dfrac{v}{c} \right) \ll 1$:
\begin{equation*}
MISSING MATRIXGLEICHUNG
\end{equation*}


\subsubsection*{Geschwindigkeitsaddition}

MISSING GRAFIK

Boosts nur in x-Richtung
\begin{equation}
MISSING VEKTORGLEICHUNG
\end{equation}

Die Rapiditäten werden linear addiert!

\begin{equation}
\boxed{
\Psi = \Psi_1 + \Psi_2}
\end{equation}

\begin{equation*}
\dfrac{v}{c} = \tanh (\Psi) = \tanh(\Psi_1 \Psi_2) = \dfrac{\tanh \Psi_1 \tanh\Psi_2}{1+\tanh\Psi_1\tanh\Psi} = \dfrac{\frac{v_1}{c}+ \frac{v_2}{c}}{1 + \frac{v_1v_2}{c^2}}
\end{equation*}


\begin{equation}
\boxed{
v = \dfrac{v_1+v_2}{1+\frac{v_1v_2}{c^2}}}
\end{equation}

\begin{equation*}
v \approx \begin{cases} v_1 + v_2 &    v_1,v_2 \ll c \\
\rightarrow c   &  v_1 \rightarrow c \text{ oder } v_2 \rightarrow c \end{cases}
\end{equation*}

MISSING GRAFIK


\subsubsection*{Allgemeiner (rotationsfreier) Lorentzboost}

\begin{equation}
\Lambda(\vec{v}) = MISSINGMATRIX
\end{equation}

\subsubsection*{Geschwindigkeitsaddition für $\vec{v_1}$, $\vec{v_2}$ beliebig}

\begin{equation*}
x^4 = \Lambda(\vec{v}_2)x^\prime = \Lambda(\vec{v}_2) \Lambda(\vec{v}_1) x = \Lambda(\vec{v})x
\end{equation*}

\begin{equation*}
\begin{aligned}
\vec{v} &= \dfrac{\gamma_1 \gamma_2}{\gamma} \left[ \vec{v}_1 + \dfrac{\vec{v}_2}{\gamma_1} - \vec{v}_1  \dfrac{\vec{v}_1 \vec{v}_2}{\vec{v}_1^2} \left( \dfrac{1}{\gamma_1} -1 \right) \right] \\
&= \dfrac{\vec{v}_1+ \vec{v}_{2 \parallel}+\vec{v}_{2\perp}\sqrt{1-\frac{v1^2}{c^2}}}{1+\frac{\vec{v}_1\vec{v}_2}{c^2}}
\end{aligned}
\end{equation*}
mit $\vec{v}_{2\parallel} \parallel \vec{v}_1$ und $\vec{v}_{2\perp} \perp \vec{v}_1$

\begin{equation*}
r_i = \sqrt{1-\frac{\vec{v}_i^2}{c^2}}
\end{equation*}

\begin{equation}
r = \Lambda^0_0(\vec{v}) = \gamma_1\gamma_2 \left(1+ \frac{\vec{v}_1 \vec{v}_2}{c^2} \right)
\end{equation}

%%----------------------------------------28.10.-----------------------------------------



Größen, die unter LTen $\Lambda$ wie $x=(x^\alpha)$ transformieren
\begin{align*}
IS &\xrightarrow{\Lambda} IS^\prime \\
a &\rightarrow a^\prime = \Lambda a \\
a ^\alpha &\rightarrow a^{\prime \alpha} = \Lambda^\alpha_\beta a ^\beta
\end{align*}
bezeichnet man allgemein als 4-Vektoren, oder Lorentzvektoren

\subsection*{Längenkontraktion und Zeitdilatation}
Maßstab mit ''\underline{Eigenlänge}'' $l_0$ den in seinem Ruhesystem $IS^\prime$ entlang der $x$-Achse liegt. Weltlinie des Maßstabs im $IS^\prime$:
\begin{equation*}
x_1^\prime = MISSINGVEKTOR
\end{equation*}
\begin{equation*}
x_2^\prime = MISSINGVEKTOR
\end{equation*}

$IS^\prime$ bewege sich relativ zu $IS$ mit Geschwindigkeit $v$ (in $x$-Richtung)\\
Welche Länge $l$ misst man für den Maßstab im $IS$? \\
Längenmessung erfolgt zu fester Zeit $t=0$.\\
Für Anfangspunkt:
\begin{equation*}
MISSING MATRIXGLEICHUNG
\end{equation*}
Für Endpunkt:
\begin{equation*}
MISSING MATRIXGLEICHUNG
\end{equation*}
Längenkontraktion:
\begin{equation}
\Rightarrow \boxed{l = \underbrace{\sqrt{1-\frac{v^2}{c^2}}}_{<1} l_0}
\end{equation}
Längenkontraktion nur in Bewegungsrichtung
\begin{equation}
l_\parallel = \dfrac{1}{\gamma} l_{0\parallel} \text{ , } l_\perp = l_{0 \perp}
\end{equation}


\subsubsection*{Uhrenvergleich}
Uhr, die im $IS^\prime$ im Ursprung ruht.\\
$IS^\prime$ bewege sich relativ zu IS mit Geschwindigkeit $v$\\
In $IS$ 2 synchronisierte Uhren:\\
eine bei $x_1=0$ und eine bei $x_2 = v t_2$\\
Die Uhr fliegt bei $x_1 = 0$ vorbei

MISSING GRAFIK

\begin{equation*}
MISSING MATRIXGLEICHUNG
\end{equation*}

Zum Zeitpunkt $t_2$ passiert die bewegte Uhr den Beobachter bei $vt_2$
\begin{equation*}
MISSING MATRIXGLEICHUNG
\end{equation*}

\begin{equation*}
\boxed{t= \dfrac{t_0}{\sqrt{1-\frac{v^2}{c^2}}}}
\end{equation*}
$t_0$ ... Zeitspanne in $IS^\prime$\\
$t$ ... Zeitspanne in $IS$


\subsection*{Eigenzeit}
Welche Uhrzeit  $\tau$ eines Teilchens bestimmen, des sich mit $\vec{v}(t)$ in $IS$ bewegt. Betrachten des Teilchens zum Zeitpunkt $t$ in einem $IS^\prime$ das sich mmit Geschwindigkeit $\vec{v}_0 = \vec{v}(t)$ gegenüber $IS$ bewegt. $\rightarrow$ $IS^\prime$ ist das ''momentane Ruheintervall'' des Teilchens.
\begin{equation*}
\Rightarrow \vec{v}^\prime \approx 0 \text{ in Zeitintervall } [t^\prime , t^\prime + dt^\prime]
\end{equation*}

\begin{equation}
\Rightarrow d\tau = dt^\prime = \sqrt{1 - \frac{v_0^2}{c^2}} dt = \sqrt{1 - \frac{v(t)^2}{c}} dt
\end{equation}
''Aufsummation'' über alle infinitesimalen Zeitintervalle


Anzeige einer mit $\vec{t}$ bewegten Uhr:
\begin{equation}
\boxed{
\tau = \int^{t_2}_{t_1} dt \sqrt{1-\frac{\vec{v}(t)^2}{c^2}}}
\end{equation}
$\tau$ ... Eigenzeit

$\tau$ hängt nicht von $IS$ ab.\\
$ds^2$ invariant unter LTen.
\begin{equation}
d\tau^2 = \dfrac{ds^2}{c^2} \Rightarrow \text{invariant unter LTen}
\end{equation}


\subsection*{Relativistische Mechanik}
Relativistisch bewegtes Teilchen unter Einfluss einer Kraft; suchen Bewegungsgleichung\\
In \underline{momentanen Ruhesystem} des Teilchens zum Zeitpunkt $t$ gilt die Newton'sche Bewegungsgleichung:
\begin{equation}\label{glg:1_newtonBewegungsgleichung}
\underbrace{m}_{\mathclap{\text{Ruhemasse}}} \dfrac{d\vec{v}^\prime(t^\prime)}{dt^2} = \underbrace{\vec{F}_N}_{\mathclap{\text{Newton'sche Kraft}}} \text{ in } IS^\prime
\end{equation}
(Unabhängig von $IS$)


versuchen 4-dimensionales Analogon von \ref{glg:1_newtonBewegungsgleichung} zu finden, das sich im momentanen Ruhesystem auf \ref{glg:1_newtonBewegungsgleichung} reduziert.

\begin{tabular}{l | p{8cm}}
Nichtrelativistisch: & Relativistisch \\ \hline
$\vec{v}(t)= \dfrac{d\vec{x}(t)}{dt}$ & Vierergeschwindigkeit \\
Räumliche Rotation &  \begin{equation}
u^\alpha(\tau) = \dfrac{dx^\alpha(\tau)}{d\tau}
\end{equation} \\
$\vec{v}(t) \xrightarrow{\hat{R}} \vec{v}^\prime(t) = \hat{R}\vec{v}(t)$ & $\tau$ ... Eigenzeit, unabhängig vom Bezugssystem\\
& Lorentz-Transformation \\
& \begin{equation}
u^\alpha (\tau) \xrightarrow{\Lambda} u^{\prime\alpha}(\tau) = \Lambda^\alpha_\beta u^\beta(\tau)
\end{equation}\\
& $d\tau = \dfrac{dt}{\gamma}$\\
& $\Rightarrow \left(u^\alpha (\tau) \right) = \gamma \left(\dfrac{dx^\alpha}{dt} \right) = \gamma(c, \vec{v}) $
\end{tabular}

$d\tau = \dfrac{ds}{c}$ invariant unter Lten \\
$dx^\alpha$ 4-Vektor \\
$\rightarrow u^\alpha = \dfrac{dx^\alpha}{d\tau}$ 4-Vektor

\subsubsection*{relativistische Verallgemeinerung von \ref{glg:1_newtonBewegungsgleichung}}

\begin{equation}
m \dfrac{du^\alpha}{d\tau} = F^\alpha
\end{equation}
Forminvariant (kovariant) unter LTen.

\begin{equation}
m \dfrac{du^\alpha}{d\tau} = F^\alpha \xrightarrow[IS \rightarrow IS^\prime]{\text{LT}} m \dfrac{du^{\prime \alpha}}{d\tau} = F^{\prime \alpha}
\end{equation}
Sofern die verallgemeinerte Kraft $F = (F^\alpha)$ wie ein 4-Vektor transformiert

\begin{equation}
F^\alpha \xrightarrow[IS \rightarrow IS^\prime]{\text{LT } \Lambda} F^{\prime \alpha} = \Lambda^\alpha_\beta F^\beta
\end{equation}
$F^\alpha$ ... Minkowskikraft


Im momentanen Ruhesystem $IS^\prime (\vec{v}^\prime = 0, \gamma = 1)$

\begin{equation}
m \left( \dfrac{du^\alpha(\tau)}{d\tau} \right) = m \gamma^\prime \dfrac{d}{dt^\prime} (c,\vec{v}^\prime) = m \left( 0, \dfrac{d\vec{v}^\prime}{dt^\prime} \right) 
\end{equation}


\begin{equation}
\left(F^{\prime\alpha} \right) = \left(F^{\prime0}, \vec{F}^\prime \right) = \left( 0, \vec{F}_N \right)
\end{equation}

\begin{equation*}
m \dfrac{du^\alpha}{d\tau} = F^\alpha \xrightarrow{\vec{v}^\prime = 0} m \dfrac{d\vec{v}^\prime(t^\prime)}{d\tau^\prime} = \vec{F}_N
\end{equation*}

Minkowskikraft in $IS$, wenn in $IS^\prime$ bekannt.
\begin{equation*}
IS^\prime \xrightarrow{\Lambda(-\vec{v})} IS
\end{equation*}
\begin{equation}
F^\alpha = \Lambda^\alpha_\beta(-\vec{v}) F^{\prime \beta}
\end{equation}

Für $\vec{v}= v^1\vec{e}_1$:
\begin{equation}
MISSING GLEICHUNG
\end{equation}

Für $\vec{v}$ beliebig
\begin{equation}
MISSING GLEICHUNG
\end{equation}

%%----------------------------------------04.11.-----------------------------------------
\newpage
MISSING EINHEIT \\
MISSING EINHEIT \\
MISSING EINHEIT \\
MISSING EINHEIT \\
MISSING EINHEIT \\
MISSING EINHEIT \\
MISSING EINHEIT \\
MISSING EINHEIT \\
MISSING EINHEIT \\
MISSING EINHEIT \\
MISSING EINHEIT \\
MISSING EINHEIT
\subsection*{Energie und Impuls}
Nicht relativistisch:
\begin{equation*}
\dfrac{d}{dt} \vec{p}_N = \vec{F}_N
\end{equation*}

\begin{equation}
p = p^\alpha := mu = \left( \gamma mc, \gamma m \vec{v} \right) = \left( \dfrac{E}{c}, \vec{p} \right)
\end{equation}

Relativistische Energie und Impuls:
\begin{equation}
E = \gamma m c^2 = \dfrac{mc^2}{\sqrt{1- \frac{v^2}{c^2}}} , \qquad \vec{p} = \gamma \vec{v} m = \dfrac{m \vec{v}}{\sqrt{1-\frac{v^2}{c^2}}}
\end{equation}

Nicht relativistischer Grenzfall: $\left( | \vec{v} | \ll c, \qquad \gamma \rightarrow 1 \right)$:
\begin{equation*}
\vec{p} \rightarrow \vec{p}_N = m \vec{v}
\end{equation*}

\begin{equation}
m \dfrac{du^0}{d\tau} = \dfrac{dp^0}{d\tau} = F^0 \xrightarrow{\gamma \rightarrow 1} \dfrac{dp^0}{dt} = \dfrac{1}{c} \ubtext{\dfrac{dE}{dt}}{zeitliche Änderung der kinetischen Energie in n.r. Mechanik} = \dfrac{\vec{v} F_N}{c}
\end{equation}

\subsubsection*{Relativistische Energie-Impuls Beziehung}
\begin{equation*}
p^\alpha p_\alpha = \dfrac{E^2}{c^2} - \vec{p} = \gamma^2 \left( m^2 c^2 - m^2 \vec{v}^2 \right) = m^2 c^2 \dfrac{1- \frac{\vec{v}^2}{c^2}}{1- \frac{\vec{v}^2}{c^2}} = m^2 c^2
\end{equation*}

\begin{equation}
\boxed{E^2 = m^2c^4 + c^2 \vec{p}^2}
\end{equation}

Masselose Teilchen ($m=0$): $E = c |\vec{p}|$

\begin{equation}
\begin{aligned}
E &= E_0 + E_{kin} \\
E_0 &= mc^2 \qquad &... \text{ Ruheenergie}\\
E_{kin} &= E -E_0 \qquad &... \text{ kinetische Energie}
\end{aligned}
\end{equation}

Für abgeschlossene Systeme ist Gesamtenergie $E$ und Gesamtimpuls $\vec{P}$ erhalten.
\begin{equation}
M c^2 = \sqrt{E^2 -c^2P^2}
\end{equation}
$M$ $...$ invariante Masse des Gesamtsystems, hängt vom Impuls der einzelnen Teilchen ab.

Atomkern mit $N$ Neutronen und $Z$ Protonen
\begin{equation}
\left( N m_n + Z m_p \right) c^2 = \ubtext{M_k}{Kernmasse} c^2 + \ubtext{\Delta E}{Bindungsenergie}
\end{equation}


MISSING EINHEIT \\
MISSING EINHEIT \\
MISSING EINHEIT \\
MISSING EINHEIT \\
MISSING EINHEIT \\
MISSING EINHEIT \\
MISSING EINHEIT \\
MISSING EINHEIT \\
MISSING EINHEIT \\
MISSING EINHEIT \\
MISSING EINHEIT \\
MISSING EINHEIT
Ausbesserung Beweis:
\begin{equation*}
\Lambda^T \eta \Lambda = \eta  \Rightarrow \Lambda \eta \Lambda^T = \eta
\end{equation*}

\begin{equation*}
\Lambda^T \Lambda = 1
\end{equation*}

\begin{align*}
\left( \eta^\prime \right)_{\alpha \beta} &= \left( \Lambda^T \eta \Lambda \right)_{\alpha \beta} = \eta_{\alpha \beta} = \left( \Lambda \eta \Lambda^T \right)_{\alpha \beta} = \Lambda_\alpha^\gamma \eta_{\gamma \delta} \left( \Lambda^T \right)^\delta_\beta\\
&= \Lambda_\alpha^\gamma \Lambda_\beta^\delta \eta_{\gamma \delta}
\end{align*}
MISSING EINHEIT \\
MISSING EINHEIT \\
MISSING EINHEIT \\
MISSING EINHEIT \\
MISSING EINHEIT \\
MISSING EINHEIT \\
MISSING EINHEIT \\
MISSING EINHEIT \\
MISSING EINHEIT \\
MISSING EINHEIT \\
MISSING EINHEIT \\
MISSING EINHEIT

\newpage

\setcounter{equation}{78}

%%----------------------------------------11.11.-----------------------------------------

\subsection*{Elektrodynamik}
\begin{equation*}
m \dfrac{du^\alpha}{d\tau} = F^\alpha
\end{equation*}


\subsubsection*{Maxwell Gleichungen}

\begin{empheq}[box=\fbox]{align}
\vec{\nabla} \cdot \vec{E}  &= 4 \pi \rho_{el} \qquad && \vec{\nabla} \times \vec{B} = \dfrac{4\pi}{c} \vec{j} + \dfrac{1}{c} \dfrac{\partial \vec{E}}{\partial t} \\
\vec{\nabla} \times \vec{E} &= - \dfrac{1}{2} \dfrac{\partial \vec{B}}{\partial t} \qquad && \vec{\nabla} \cdot \vec{B} = 0
\end{empheq}

Kontinuitätsgleichung:
\begin{equation}\label{glg:1_kontinuitaetsgleichungStrom}
\partial_\alpha j^\alpha = 0
\end{equation}
\begin{equation}
\underbrace{\left( j^\alpha \right)}_{\mathclap{\text{kontra 4-Vektor}}} = \left( c \rho_{el} \cdot \vec{j} \right)
\end{equation}


Aus \ref{glg:1_kontinuitaetsgleichungStrom} folgt Ladungserhaltung 

\begin{equation*}
\partial_t \int_{\mathbb{R}^3}d^3r j^0 = \int_{\mathbb{R}^3}d^3r \vec{\nabla} \cdot \vec{j} = \int_{\partial \mathbb{R}^3} d\vec{A} \cdot \vec{j} = 0
\end{equation*}

$Q ... $ Lorentzvektor


Feldstärketensor:
\begin{equation}
F^{\alpha \beta} =  \begin{pmatrix}
0 & -E_x & -E_y & -E_z \\
E_x & 0 & -B_z & B_y \\
E_y & B_z & 0 & -B_x \\
E_z & -B_y & B_x & 0
\end{pmatrix}
\end{equation}

\begin{empheq}[box=\fbox]{align}
\partial_\alpha F^{\alpha \beta} &= \dfrac{4\pi}{c} j^\beta \label{glg:1_partialFeldstaerketensor}\\
\epsilon^{\alpha \beta \gamma \delta} \partial_\beta F_{\gamma \delta} &= 0
\end{empheq}

Aus \ref{glg:1_partialFeldstaerketensor} folgt $F^{\alpha \beta}$ kontravarianter Lorentzvektor 2. Stufe


\subsection*{Elektrodynamische Potentiale}
\begin{equation}
F^{\alpha \beta} = \partial^\alpha A^\beta - \partial^\beta A^\alpha
\end{equation}

\begin{equation*}
\left( A^\alpha \right) = \left( \Phi, \vec{A} \right)
\end{equation*}

$\left( A^\alpha \right) $ bur bis auf 4-divergenz festgelegt, das heißt $F^{\alpha \beta}$ unverändert unter \underline{Eichtransformation}
\begin{equation}
A^{\alpha\prime} \rightarrow A^\alpha + \partial^\alpha \chi
\end{equation}

$\chi (x) ... $ skalares Feld

\subsubsection*{Lorentzeichung}

\begin{empheq}[box=\fbox]{align}
\partial_\alpha A^\alpha &= 0 \\
\square A^\alpha &= \dfrac{4 \pi}{c} j^\alpha
\end{empheq}

\subsection*{Kopplung eines gebundenen Teilchens an ein elektromagnetisches Feld}
\begin{equation}
m \dfrac{du^\alpha}{d\tau} = F^\alpha = \dfrac{q}{c} \underbrace{F^{\alpha \beta}}_{\mathclap{\text{relativistische Lorentzkraft}}} u_\beta
\end{equation}


Vergleich mit Lagrangefunktion:
\begin{equation*}
L = \underbrace{- mc \sqrt{u^\alpha u_\alpha}}_{\mathclap{\text{freies relativistisches Teilchen}}} - \frac{q}{c} A^\beta u_\beta
\end{equation*}

Für räumlichen Anteil:
\begin{equation}
\dfrac{d}{dt} \dfrac{m \vec{v}}{\sqrt{1 - \frac{v^2}{c^2}}} = \ubtext{q \left( \vec{E} + \frac{\vec{v}}{c} \times \vec{B} \right)}{Lorentzkraft}
\end{equation}

\subsection*{Energie-Impulstensor}

\begin{equation}
T^{\alpha \beta}_{em} = \dfrac{1}{4 \pi} \left( F^\alpha_\gamma F^{\gamma \beta} + \dfrac{1}{4} \eta^{\alpha \beta} F_{\gamma \delta F^{\gamma \delta}} \right)
\end{equation}

$00$-Komponente $...$ Energiedichte
\begin{equation}
u_{em} = T_{em}^{00} = \dfrac{1}{4 \pi} \left( \vec{E}^2 + \vec{B}^2 \right)
\end{equation}

$0i$-Komponente $...$ Energiestromdichte ($\vec{S} ...$ Poyntingvektor)

\begin{equation}
\vec{S} = c \sum_{i = 1}^3 T_{em}^{0i} \vec{e}_i = \dfrac{c}{4 \pi} \left( \vec{E}^2 + \vec{B}^2 \right)
\end{equation}

$T^{j0} ...$ Impulsdichten und $T^{ji} ...$ Impulsstromdichten analog 

Aus Maxwellgleichungen folgt:
\begin{equation}\label{glg:1_partialEnergieImpulsTensor}
\partial_\alpha T^{\alpha \beta}_{em} = - \dfrac{1}{c} F^{\beta \gamma}j_\gamma
\end{equation}


Im Ladungsfreien Raum ($j_\gamma = 0$)
\begin{equation}
\partial_\alpha T_{em}^{\alpha \beta} = 0
\end{equation}

Für abgeschlossenes System (keine Kopplung an Strom) folgt \\4-Impulserhaltung:
\begin{equation}
P^\beta_{em} = \int_{\mathbb{R}^3} d^3r T_{em}^{0\beta} = konst.
\end{equation}

Für räumliche Komponenten folgt aus \ref{glg:1_partialEnergieImpulsTensor}
\begin{equation}
\dfrac{1}{c}F^{i \gamma} j_\gamma \vec{e}_i = \rho \vec{E} + \dfrac{1}{c} \vec{j} + \vec{B} = \underbrace{\vec{f}}_{\mathclap{\text{Lorentzkraftdichte}}}
\end{equation}

\ref{glg:1_partialEnergieImpulsTensor} kann geschrieben werden als
\begin{equation}\label{glg:1_KraftdichteEMFeld}
\partial_\alpha T^{\alpha \beta}_{em} = - \underbrace{f^\beta}_{\mathclap{\text{Kraftdichte die e.m. Feld auf Stromverteilung ausübt}}}
\end{equation}
$\Rightarrow$ Austausch von Energie auf Impuls zwischen e.m. Feld und Strom



\subsection*{Relativistische Hydrodynamik}
ideale Flüssigkeiten

$\rho(\vec{r}, t)$ $...$ Massendichte\\
$\vec{v}(\vec{r}, t) = v^i \vec{e}_i$ $...$ Geschwindigkeitsfeld\\
$P(\vec{r}, t)$ $...$ isotroper Druck (Skalar, in alle Richtungen gleich) \\
Viskosität (innere Reibung) vernachlässigbar \\
Massenelement $\delta m$ mit Volumen $\delta V$
\begin{align}
\Delta m \dfrac{d\vec{v}}{dt} &= \Delta \vec{F_N} \nonumber \\ 
\underbrace{\dfrac{\Delta m}{\Delta V}}_{\mathclap{\rho(\vec{r}, t)}} \dfrac{d\vec{v}(\vec{r}, t)}{dt} &= \underbrace{\dfrac{\Delta \vec{F_N}}{\Delta V}}_{\mathclap{\vec{f}_N(\vec{r}, t)}}
\end{align}

$\vec{f}_N(\vec{r}, t) = - \underbrace{ \vec{\nabla}  P (\vec{r}, t)}_{\text{Druckgradient}} + \underbrace{\vec{f}_0 (\vec{r}, t)}_{\text{äußere Kraft (z.B. Gravitation)}}$ $...$ Newton'sche Kraftdichte

\begin{equation}
\begin{aligned}
d \vec{v}(\vec{r}, t) &= dt \dfrac{d\vec{v}}{dt} +dx \dfrac{d\vec{v}}{dx} +dy \dfrac{d\vec{v}}{dy} + dz \dfrac{d\vec{v}}{dz}\\
&= \left(\dfrac{\partial \vec{v}}{\partial t} dt + \ubtext{ \left( d\vec{r} \cdot \vec{\nabla} \right)}{$d\vec{v} dt$}  \vec{v} \right) \\
&= \left(\dfrac{\partial \vec{v}}{\partial t} + \left( d\vec{v} \cdot \vec{\nabla} \right)  \vec{v} \right) dt \\
\end{aligned}
\end{equation}

Eulergleichung:
\begin{equation}\label{glg:1_eulergleichung}
\rho \left[ \dfrac{\partial \vec{v}}{\partial t} + \left( \vec{v} \cdot \vec{\nabla} \right) \vec{v} \right] = - \vec{\nabla} P + \vec{f}_0
\end{equation}

Kontinuitätsgleichung:
\begin{equation}
\dfrac{d \rho}{dt} + \vec{\nabla} \cdot \left( \rho \vec{v} \right) = 0
\label{glg:1_kontinuitaetsgleichung}
\end{equation}

\ref{glg:1_eulergleichung} + \ref{glg:1_kontinuitaetsgleichung} nichtrelativistische Feldgleichung einer idealen Flüssigkeit


5 unbekannte Felder $\rho$, $P$, $v^i$, aber nur 4 Gleichungen.\\
Brauche Beziehung zwischen $\rho$ und $P$.\\
\underline{Beispiele:} 
\renewcommand{\labelitemi}{}
\begin{itemize}
\item $\rho = konst.$ $..$ inkompressible Flüssigkeit
\item $\dfrac{P}{\rho} = konst.$ $...$ ideales Gas bei festem T 
\end{itemize}

\subsubsection*{Relativistische Verallgemeinerung?}
\begin{equation*}
v^i (\vec{r}, t) \rightarrow u^\alpha(x), x = (x^0, x^1, x^2, x^3)
\end{equation*}
linke Seite $\rightarrow \vec{v}$ quadratisch $\rightarrow$ Ansatz:
\begin{equation}
M^{\alpha \beta} = \rho u^\alpha u^\beta
\end{equation}
$M^{\alpha \beta} \rightarrow$ Kontravarianter Term 2. Stufe 

$\rho(x)$ $...$ Massendichte $\rightarrow$ \underline{Lorentz-Skalar}

\begin{equation*}
\rho(x) = \rho^\prime(x^\prime) = \dfrac{\Delta m}{\Delta V} = \dfrac{\text{Ruhemasse}}{\text{Eigenvolumen}}
\end{equation*}
$\rho^\prime(x^\prime)$ $...$ Dichte des Flüssigkeitselements $\Delta V$ am Ort $x$ im momentanen Ruhesystem $IS^\prime$

\begin{equation}
\left( u^\alpha \right) = \gamma \left( c, \vec{v} \right)
\end{equation}

\begin{equation}
\left( M^{\alpha \beta} \right) = \rho \gamma^2 c^2 \begin{pmatrix}
1 & \frac{v^1}{c} & \frac{v^2}{c} & \frac{v^3}{c} \\
\frac{v^1}{c} \\
\frac{v^2}{c} & & \dfrac{\vec{v}\vec{v}^T}{c^2} \\
\frac{v^3}{c} \\
\end{pmatrix}
\end{equation}

Analogon zur Ladungsdichte $\rho_{em}$ in Elektrodynamik
\begin{equation}
\tilde{\rho} = \dfrac{M^{00}}{c^2} = \gamma^2 \rho = \dfrac{\rho}{1- \frac{v^2}{c^2}}
\end{equation}
$\tilde{\rho}$ $...$ Energie Massendichte\\
$\tilde{\rho}$ transformiert wie $00$-Komponente von Tensor

\begin{equation}\label{glg:1_partial0KomponenteMassenTensor}
\partial_\beta M^{0 \beta} = c \left[ \partial_t \tilde{\rho} + \partial_k \left(\tilde{\rho} v^k \right) \right]
\end{equation}
\begin{equation}\label{glg:1_partialIKomponenteMassenTensor}
\begin{aligned}
\partial_\beta M^{i \beta} &=  \partial_t \left( \tilde{\rho} v^i \right) + \partial_k \left(\tilde{\rho} v^i v^k \right) \\
&= \tilde{\rho} \left( \partial_t v^i + v^k \partial_k v^i \right) + \ubtext{ v^i \left[ \partial_t \tilde{\rho} + \partial_k \left(\tilde{\rho} v^k \right) \right]}{$=0$ wegen Kontinuitätsgleichung}
\end{aligned}
\end{equation}

\underline{Für $v \ll c$:}\\
\ref{glg:1_partial0KomponenteMassenTensor} $\rightarrow$ linke Seite von \ref{glg:1_kontinuitaetsgleichung}\\
\ref{glg:1_partialIKomponenteMassenTensor} $\rightarrow$ linke Seite von \ref{glg:1_eulergleichung}


\underline{Kräftefreier Fall:} ($P=0$, $f_0=0$)\\
kovariante Verallgemeinerung der Strömungsgleichung
\begin{equation}\label{glg:1_differentiellerErhaltungssatzImpulsdichte}
\partial_\beta M^{\alpha \beta} = 0
\end{equation} 

Kontinuitätsgleichung (für Energiestrom)
\begin{equation}
\partial_\beta M^{0 \beta} = 0
\end{equation}

Eulergleichung (Kontinuitätsgleichung für Impulsstrom)
\begin{equation}
\partial_\beta M^{i \beta} = 0
\end{equation}

\ref{glg:1_differentiellerErhaltungssatzImpulsdichte} differenzieller Erhaltungssatz für 4-Impulsdichte $\Rightarrow$ 4-Impulserhaltung



%%----------------------------------------18.11.-----------------------------------------


Nicht relativistisch, anisotropes Medium (Druck kann Richtungsabhängig sein).\\
Kraft auf ein gerichtetes Flüssigkeitselement

MISSING GRAFIK

\begin{equation}
d \vec{F} = \hat{P} d \vec{A} = \sum_{i =1}^3 d F^i \vec{e}^i \qquad \text{mit } dF^i = \sum_{j=1}^3 P^{i j} dA^j
\end{equation}

Im momentanen Ruhesystem $IS^\prime$ des Flüssigkeitselements an Ort $\vec{x}$ zum Zeitpunkt $t$ und isotroper Druckverteilung:
\begin{equation}
\left(P^{\prime ij} \right) = \begin{pmatrix}
P &0 &0 \\ 0 & P & 0\\ 0&0 & P 
\end{pmatrix}
\end{equation}

relativistisch verallgemeinerter 4-Tensor, in $IS^\prime$ gelten \ref{glg:1_eulergleichung} $\rightarrow$ auf rechter Seite steht.
\begin{equation}
\partial^\prime_i P = \partial^\prime_j P^{\prime i j}
\end{equation}

4-Divergenz des Drucktensors auf rechter Seitevon \ref{glg:1_differentiellerErhaltungssatzImpulsdichte}.
\begin{equation}
\left( \partial^\prime_\beta P^{\prime \alpha \beta} \right) = \begin{pmatrix}
0 &  \partial^\prime_i P
\end{pmatrix}
\end{equation}
Druck in $IS^\prime$ ... ''Eigendruck''\\
Druck in $IS$, in dem Flüssigkeitselement Geschwindigkeit $\vec{v}$ aufweist $\rightarrow$ Lorentzboost mit $- \vec{v}$

\begin{equation}
P^{\alpha \beta} = \Lambda^\alpha_\gamma \Lambda^\beta_\delta P^{\prime \gamma \delta} = P \left( \dfrac{u^\alpha u^\beta}{c^2} - \eta^{\alpha \beta} \right)
\end{equation}

Eulergleichung + Kontinuitätsgleichung relativistisch (für $\vec{f}_0 = 0$)

\begin{equation}
\partial_\beta M^{\alpha \beta} + \partial_\beta P^{\alpha \beta} = 0
\end{equation}


\subsection*{Energie Impulstensor}
\begin{equation}
T^{\alpha \beta} = M ^{\alpha \beta} + P ^{\alpha \beta} = \left( \rho + \dfrac{P}{c^2} \right) u^\alpha u^\beta - \eta^{\alpha \beta} P
\end{equation}

Eulergleichung + Kontinuitätsgleichung mit Energie Impulstensor:
\begin{equation}
\partial_\beta T^{\alpha \beta} = 0
\end{equation}

Mit äußeren Kräften ($\vec{f}_0 \rightarrow$ Minkowskikraft $f^\alpha$):

Relativistische Grundgleichung der Hydrodynamik:
\begin{equation}
\boxed{
\partial_\beta T^{\alpha \beta} = f^\alpha}
\end{equation}




Für abgeschlossens System ($f^\alpha = 0$):
\begin{equation}
\partial_\beta T^{\alpha \beta} = 0
\end{equation}
Kontinuitätsgleichung für Energie- und Impulsdichten

\begin{equation}
\dfrac{\partial}{\partial (c t)} \int_{\mathbb{R}^3} d^3 r T^{\alpha 0} = -  \int_{\mathbb{R}^3} d^3 r \partial_i T^{\alpha i} \overset{\text{Gauss}}{=} -  \int_{\partial (\mathbb{R}^3)} dS_i T^{\alpha i} = 0
\end{equation}

$T^{\alpha i}(x) \xrightarrow{| \vec{x} | \rightarrow \infty} 0$ genügend schnell\\
$\Rightarrow$ 4- Impulserhaltung
\begin{equation}
P^\alpha = \dfrac{1}{c}  \int_{\mathbb{R}^3} d^3r T^{\alpha 0} = konst \qquad \alpha = 0,1,2,3
\end{equation}



\begin{equation}
\begin{aligned}
&T^{00} \rightarrow \text{ Energiedichte } \Rightarrow  \int_{\mathbb{R}^3} d^3r T^{00} \text{ Energie } \\
&\dfrac{T^{i0}}{c} \rightarrow \text{ Impulsdichte } \Rightarrow  \int_{\mathbb{R}^3} d^3r \dfrac{T^{i0}}{0} \text{ Impuls }
\end{aligned}
\end{equation}


$cP^0$ $...$ Energie\\
$P^i$ $...$ Impuls


Energie- Impulserhaltung gilt für abgeschlossne Systeme\\
geladene Flüssigkeit, auf die e.m. Kräfte wirken
\begin{equation}
\partial_\beta T^{\alpha \beta} = \ubtext{f^\alpha}{Minkowskikraftdichte} \overset{\text{\ref{glg:1_KraftdichteEMFeld}}}{=} - \partial_\beta \ubtext{T^{\alpha \beta}_{em}}{Energie-Impulstensor des e.m. Feldes}
\end{equation}

\begin{equation}
\partial_\beta \left( T^{\alpha \beta} + T^{\alpha \beta}_{em} \right) = 0
\end{equation}

\underline{Allgemein:}

$\Rightarrow$ Energie und Impulserhaltung für Gesamtsystem aber nicht getrennt für Flüssigkeit und e.m. Feld.\\
Zusätzliche Kräfte können zu Bestandteil des Energie- Impulstensor gemacht werden, sodass in Energie- Impulstensor alle Energieformen berücksichtigt sind.

\begin{equation}
T^{\alpha \beta} = M^{\alpha \beta} + P^{\alpha \beta} + T^{\alpha \beta}_{em} + ...
\end{equation}

\begin{equation}
\partial T^{\alpha \beta} = 0, \qquad T^{\alpha \beta} = T^{\beta \alpha}
\end{equation}
Energie- Impulstensor für Gesamtsystem


\begin{tabular}{c | c}
\underline{Elektrodynamik} 
& \underline{ideale Flüssigkeit} \\
$\rho_{el}$ $...$ elektrische Ladungsdichte 
& $\tilde{\rho} = \gamma^2 \rho$ $...$ Energie- Massedichte \\
$(j^\alpha)$ $...$ $( \rho_{el}, \vec{j} )$ Strom
& $\left( T^{\alpha \beta} \right) = \begin{pmatrix}
c^2 \tilde{\rho} = c^2 \gamma^2 \rho & \cdots \\ \hdots
\end{pmatrix}$ \\
$\partial j^\alpha = 0$ $\Rightarrow$ Ladungserhaltung
& $\partial T^{\alpha \beta} = 0$ $\Rightarrow$ 4-Impuls\\
$j^\alpha$ $...$ Quelle des e.m. Felds 
& $T^{\alpha \beta}$  $...$ Quelle des Gravitationsfeldes \\
$\square A^\alpha (x) = \dfrac{4 \pi}{c} j^\alpha (x)$ & 
 $\square g^{\alpha \beta}(x) \sim G T^{\alpha \beta}(x)$ \\
$...$ inhomogene Maxwellgleichung & $...$ Einstein'sche Feldgleichung \\
& $T^{\alpha \beta}$ enthält \underline{alle} Energieformen \\&
Wie sieht $g_{\alpha \beta}(x)$ aus? \\
& Enthält $T^{\alpha \beta}$ auch die Energie des \\& Gravitationsfeldes?
\end{tabular}


\newpage
\section{Grundlagen der \\Allgemeinen Relativitätstheorie}

\renewcommand{\labelitemi}{$\bullet$}
\begin{itemize}
\item relativistische Gleichungen $\xrightarrow{|\vec{v}| \ll c}$ nichtrelativistische Gleichungen

\item Kovarianzprinzip:relativistische Gleichungen lassen sich im kovarianter Form, das heißt als gleichungen für Lorentz-Tensoren formulieren, die forminvariant unter Lorentz-Transformationen sind
\end{itemize}

Für relativistische Formulierung der Gravitation
\begin{itemize}
\item \underline{Äquivalenzprinzip}\\
\underline{Schwache Form}: träge und schwere Masse sind gleich\\
\underline{Starke Form} (Einstein'sches Äquivalenzprinzip):\\
Für ein beliebiges Gravitationsfeld ist es in jedem Raum- Zeitpunkt $X$ möglich, ein \underline{lokales Inertialsystem} zu finden, sodass in einer hinreichend kleinen Umgebung des Punktes $X$ die Gesetze der speziellen Relativitätstheorie \underline{ohne Gravitation} gelten.

ART $\xrightarrow{\text{Äquivalenzprinzip}}$ SRT\\
(analog SRT $\xrightarrow{|\vec{v}| \ll c}$ NM)
\item \underline{Allgemeines Kovarianzprinzip}: Gleichungen der ART sind forminvariant (kovariant) unter allgemeinen (nicht singuläre) Koordinatentransformationen $x \rightarrow x^\prime$.
\end{itemize}

\subsection*{Äquivalenzprinzip}
Frei fallender Körper nahe der Erdoberfläche (nicht relativistisch)
\begin{equation*}
m_t \ddot{z} = -m_s g
\end{equation*}

Für $z(0) = 0$, $\dot{z}(0) = 0 $ ist die Lösung
\begin{equation}
z(t) = - \dfrac{1}{2} \dfrac{m_s}{m_t} g t^2
\end{equation}

Für Pendel bei kleinen Auslenkungen ist die Schwingungsdauer
\begin{equation*}
\left(\dfrac{T}{2 \pi} \right)^2 = \left(\dfrac{m_t}{m_s} \right) \left( \dfrac{l}{g} \right)
\end{equation*}

Konstanu von $\dfrac{m_s}{m_t}$:\\
Newton: auf $10^{-3}$ genau\\
Eötvös (1922): auf $ 5 \times 10^{-9}$ genau\\
Aktuelles Ergebnis: auf $2 \times 10^{-13}$ genau


$[m_s] = [m_t] = 1 \text{ kg}$\\
$\Rightarrow$ Wert von $G$ in \ref{glg:0_Gravitationskonstante}\\
$m_s = m_t$\\
Trägt Energie des Gravitationsfeldes auch zu $m_s$ und $m_t$ in gleichem Maße bei?




















\end{document}